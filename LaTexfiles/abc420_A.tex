\documentclass{article}
\usepackage{graphicx} % Required for inserting images
\usepackage{kotex}
\usepackage{listings}
\usepackage{xcolor}
\usepackage{hyperref}

\title{Atcoder ABC420 A번 풀이}

\begin{document}

\definecolor{backcolour}{rgb}{0.95,0.95,0.92}
\definecolor{codegreen}{rgb}{0,0.6,0}
\definecolor{myred1}{rgb}{255, 0, 0}


% Define a custom style
\lstdefinestyle{myStyle}{
    backgroundcolor=\color{backcolour},   
    commentstyle=\color{codegreen},
    basicstyle=\ttfamily\footnotesize,
    breakatwhitespace=false,         
    breaklines=true,                 
    keepspaces=true,                 
    numbers=left,       
    numbersep=5pt,                  
    showspaces=false,                
    showstringspaces=false,
    showtabs=false,                  
    tabsize=2,
}

\hypersetup{
    colorlinks=true,
    linkcolor=blue,
    filecolor=magenta,      
    urlcolor=cyan,
}

\maketitle

\noindent 문제 링크\\
\href{https://atcoder.jp/contests/abc420/tasks/abc420_a}{$https://atcoder.jp/contests/abc420/tasks/abc420_a$}\\

\noindent 알고리즘\\\\
수학\\

\noindent 시간복잡도\\\\
$O(1)$\\

\noindent 풀이

\noindent $X + Y$가 $12$보다 작거나 같으면 $X + Y$월이 되므로 쉽게 문제를 해결할 수 있습니다.\\\\
하지만, $X + Y \geq 13$인 순간부터 다시 $1$월이 되므로 $1$, $13$, $25$, $...$이 같은 $1$월이 되어야 하므로 $12p + q$꼴의 값은 $q$월이 됨을 알 수 있습니다. 이를 정리하면 $(X + Y - 1) \% 12 + 1$월이 정답임을 알 수 있습니다.\\

% code block
\lstset{style=myStyle}
\begin{lstlisting}[caption=소스코드, language=Matlab]
#include <stdio.h>
#include <algorithm>
#include <cmath>
#include <vector>
#include <string.h>
#include <map>
#include <queue>
#include <unordered_map>
using namespace std;
typedef long long ll;

ll a[200005];

int main()
{
    ll t = 1;
    while(t--)
    {
        ll x,y;
        scanf("%lld %lld",&x,&y);
        printf("%lld",(x+y-1)%12+1);
    }
}

\end{lstlisting}


\end{document}
