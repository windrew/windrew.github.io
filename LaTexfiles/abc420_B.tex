\documentclass{article}
\usepackage{graphicx} % Required for inserting images
\usepackage{kotex}
\usepackage{listings}
\usepackage{xcolor}
\usepackage{hyperref}

\title{Atcoder ABC420 B번 풀이}

\begin{document}

\definecolor{backcolour}{rgb}{0.95,0.95,0.92}
\definecolor{codegreen}{rgb}{0,0.6,0}
\definecolor{myred1}{rgb}{255, 0, 0}


% Define a custom style
\lstdefinestyle{myStyle}{
    backgroundcolor=\color{backcolour},   
    commentstyle=\color{codegreen},
    basicstyle=\ttfamily\footnotesize,
    breakatwhitespace=false,         
    breaklines=true,                 
    keepspaces=true,                 
    numbers=left,       
    numbersep=5pt,                  
    showspaces=false,                
    showstringspaces=false,
    showtabs=false,                  
    tabsize=2,
}

\hypersetup{
    colorlinks=true,
    linkcolor=blue,
    filecolor=magenta,      
    urlcolor=cyan,
}

\maketitle

\noindent 문제 링크\\\\
\href{https://atcoder.jp/contests/abc420/tasks/abc420_b}{https://atcoder.jp/contests/abc420/tasks/abc420\_b}\\

\noindent 알고리즘\\\\
구현, 문자열\\

\noindent 시간복잡도\\\\
$O(NM)$\\

\noindent 풀이\\

\noindent 최대 점수를 직접 구할 필요는 없으므로 상대 점수만 고려합시다.\\
$X = 0$이거나 $Y = 0$인 상황은 모두의 점수가 증가하므로 무시할 수 있습니다. 아닌 상황은 문제의 지문대로

\begin{itemize}
    \item $X < Y$인 경우 $1$을 투표한 사람들에게 전부 $1$점을 줍시다.
    \item 아닌 경우 $0$을 투표한 사람들에게 전부 $1$점을 줍시다.
\end{itemize}

\noindent 이후, 최댓값을 구한 다음 $1$번 사람부터 순차적으로 보면서 최댓값과 같은 값을 가지는 사람들을 전부 출력해주면 됩니다. 이는 전부 for문을 활용하여 구현할 수 있습니다.

% code block
\lstset{style=myStyle}
\begin{lstlisting}[caption=소스코드, language=Matlab]
#include <stdio.h>
#include <algorithm>
#include <cmath>
#include <vector>
#include <string.h>
#include <map>
#include <queue>
#include <unordered_map>
using namespace std;
typedef long long ll;

ll score[105];
char s[105][105];

int main()
{
    ll t = 1;
    while(t--)
    {
        ll i,j,n,m,mx = -1;
        scanf("%lld %lld",&n,&m);
        for(i=1;i<=n;i++)
        {
            scanf("%s",s[i]);
        }
        for(i=0;i<m;i++)
        {
            ll x,y = 0;
            for(j=1;j<=n;j++)
            {
                if(s[j][i]-'0')y++;
            }
            x = n - y;
            if(y == 0 || x == 0)continue;
            if(y > x)
            {
                for(j=1;j<=n;j++)
                {
                    if(!(s[j][i]-'0'))score[j]++;
                }
            }
            else
            {
                for(j=1;j<=n;j++)
                {
                    if(s[j][i]-'0')score[j]++;
                }
            }
        }
        for(i=1;i<=n;i++)
        {
            if(score[i] > mx)mx = score[i];
        }
        for(i=1;i<=n;i++)
        {
            if(score[i] == mx)printf("%lld ",i);
        }
    }
}

\end{lstlisting}


\end{document}
